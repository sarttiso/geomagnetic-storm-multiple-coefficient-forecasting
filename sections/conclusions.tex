This work has demonstrated a NN architecture capable of learning reliable measures of uncertainty in its forecasts of geomagnetic storms. Learning uncertainty in NN output results in more useful probabilistic forecasts than learning uncertainty in the NN parameters, and the choice of output distribution and cost function has a large impact on the resulting reliability of the trained network. Specifically, adding regularizing terms in the likelihood cost function improves the forecast reliability by incentivizing networks to forecast more reasonable mean values rather than simply increasing forecast uncertainty.

These neural networks \deleted{are also the first to }utilize as inputs observations from \added{both} the solar disk and L1 point\replaced{, slightly improving forecast reliability and skill with respect to networks trained only with L1 inputs. }{. This provides improved uncertainty forecast and higher reliability, although, } \added{However}, storm arrival and amplitude forecasting did not substantially improve from the inclusion of these data. Thus, leveraging time series of observations of the solar disk, which are often sparse, remains an open problem, and future network architectures must be carefully designed to utilize these data sources. 