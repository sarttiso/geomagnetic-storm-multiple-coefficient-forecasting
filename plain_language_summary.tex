Geomagnetic storms are capable of damaging infrastructure like power grids and communication lines, motivating our need to forecast them. Solar phenomena are the sources of geomagnetic storms, which occur when these phenomena reach Earth as bursts of the solar wind. Decades of satellite observations of both the solar wind near the Earth as well as of the Sun itself are promising for forecasting geomagnetic storms with new algorithms known as neural networks. These are the same algorithms now used to navigate self-driving cars and for speech recognition, and various neural network architectures have already been applied to geomagnetic storm forecasting. However, their full potential remains unexplored. First, all implemented neural networks have only used measurements of the solar wind one hour upstream of the Earth or closer. While these observations are critical for understanding how a storm will progress, it is nearly impossible to predict a storm's arrival more than an hour in advance. For this reason, we include observations of the Sun itself, which reach Earth much faster than the solar wind, potentially allowing for storm forecasting several hours or even days in advance. Second, all neural networks used for geomagnetic storm forecasting to date have generated forecasts without any uncertainty, meaning that end-users (such as utilities or telecommunications companies) know little about forecast confidence. We present an architecture that generates estimates of uncertainty, and our results demonstrate that neural networks learn how confident to be in their forecasts.